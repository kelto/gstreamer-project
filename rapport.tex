\documentclass[10pt,a4paper]{article}
\usepackage{xcolor}
\usepackage[utf8]{inputenc}
\usepackage[french]{babel}
\usepackage[T1]{fontenc}
\usepackage{amsmath}
\usepackage{amsfonts}
\usepackage{amssymb}
\usepackage{listings}

\title{OIM\\ Rapport de Projet 2}
\date{}

\input{/home/kelto/Documents/cours/LaTexTemplate/templates/listings.tex }

\begin{document}
\maketitle

\part*{Ligne de commande}
\section*{Question 1}
Format avi: 
Format OGV:
Format Vorbis:
Format Theora:
\section*{Question 2}

La première commande permet de lire le fichier video.ogv avec son et vidéo.
\begin{lstlisting}
gst-launch filesrc location=video.ogv ! oggdemux name=demux demux. ! queue ! theoradec ! autovideosink demux. ! queue ! vorbisdec ! autoaudiosink 
\end{lstlisting}

Cette seconde commande permet de lire le flux vidéo du fichier video.ogv avec le flux audio du fichier cranberries.mp3 fourni
\begin{lstlisting}
gst-launch filesrc location=video.ogv ! oggdemux name=demux demux. ! queue ! theoradec ! autovideosink filesrc location=cranberries.mp3 ! queue ! mad  ! autoaudiosink
\end{lstlisting}

\section*{Question 3}

\begin{lstlisting}
gst-launch subtitleoverlay name=overlay ! autovideosink  filesrc location=video.ogv ! oggdemux name=demux demux. ! queue ! theoradec ! overlay. demux. ! queue ! vorbisdec ! autoaudiosink filesrc location=video.srt ! subparse subtitle-encoding=UTF-8 ! overlay.

\end{lstlisting}

\section*{Question 4}

\begin{lstlisting}
gst-launch subtitleoverlay name=overlay ! theoraenc ! oggmux name=muxer ! filesink location=video_soustitres_partiel.ogv  filesrc location=video.ogv ! oggdemux name=demux demux. ! queue ! theoradec ! overlay. filesrc location=cranberries.mp3 ! queue ! mad ! audioconvert ! vorbisenc ! muxer. filesrc location=video.srt ! subparse subtitle-encoding=UTF-8 ! overlay.


\end{lstlisting}

\end{document}